% Options for packages loaded elsewhere
\PassOptionsToPackage{unicode}{hyperref}
\PassOptionsToPackage{hyphens}{url}
%
\documentclass[
]{article}
\usepackage{amsmath,amssymb}
\usepackage{iftex}
\ifPDFTeX
  \usepackage[T1]{fontenc}
  \usepackage[utf8]{inputenc}
  \usepackage{textcomp} % provide euro and other symbols
\else % if luatex or xetex
  \usepackage{unicode-math} % this also loads fontspec
  \defaultfontfeatures{Scale=MatchLowercase}
  \defaultfontfeatures[\rmfamily]{Ligatures=TeX,Scale=1}
\fi
\usepackage{lmodern}
\ifPDFTeX\else
  % xetex/luatex font selection
\fi
% Use upquote if available, for straight quotes in verbatim environments
\IfFileExists{upquote.sty}{\usepackage{upquote}}{}
\IfFileExists{microtype.sty}{% use microtype if available
  \usepackage[]{microtype}
  \UseMicrotypeSet[protrusion]{basicmath} % disable protrusion for tt fonts
}{}
\makeatletter
\@ifundefined{KOMAClassName}{% if non-KOMA class
  \IfFileExists{parskip.sty}{%
    \usepackage{parskip}
  }{% else
    \setlength{\parindent}{0pt}
    \setlength{\parskip}{6pt plus 2pt minus 1pt}}
}{% if KOMA class
  \KOMAoptions{parskip=half}}
\makeatother
\usepackage{xcolor}
\usepackage[margin=1in]{geometry}
\usepackage{color}
\usepackage{fancyvrb}
\newcommand{\VerbBar}{|}
\newcommand{\VERB}{\Verb[commandchars=\\\{\}]}
\DefineVerbatimEnvironment{Highlighting}{Verbatim}{commandchars=\\\{\}}
% Add ',fontsize=\small' for more characters per line
\usepackage{framed}
\definecolor{shadecolor}{RGB}{248,248,248}
\newenvironment{Shaded}{\begin{snugshade}}{\end{snugshade}}
\newcommand{\AlertTok}[1]{\textcolor[rgb]{0.94,0.16,0.16}{#1}}
\newcommand{\AnnotationTok}[1]{\textcolor[rgb]{0.56,0.35,0.01}{\textbf{\textit{#1}}}}
\newcommand{\AttributeTok}[1]{\textcolor[rgb]{0.13,0.29,0.53}{#1}}
\newcommand{\BaseNTok}[1]{\textcolor[rgb]{0.00,0.00,0.81}{#1}}
\newcommand{\BuiltInTok}[1]{#1}
\newcommand{\CharTok}[1]{\textcolor[rgb]{0.31,0.60,0.02}{#1}}
\newcommand{\CommentTok}[1]{\textcolor[rgb]{0.56,0.35,0.01}{\textit{#1}}}
\newcommand{\CommentVarTok}[1]{\textcolor[rgb]{0.56,0.35,0.01}{\textbf{\textit{#1}}}}
\newcommand{\ConstantTok}[1]{\textcolor[rgb]{0.56,0.35,0.01}{#1}}
\newcommand{\ControlFlowTok}[1]{\textcolor[rgb]{0.13,0.29,0.53}{\textbf{#1}}}
\newcommand{\DataTypeTok}[1]{\textcolor[rgb]{0.13,0.29,0.53}{#1}}
\newcommand{\DecValTok}[1]{\textcolor[rgb]{0.00,0.00,0.81}{#1}}
\newcommand{\DocumentationTok}[1]{\textcolor[rgb]{0.56,0.35,0.01}{\textbf{\textit{#1}}}}
\newcommand{\ErrorTok}[1]{\textcolor[rgb]{0.64,0.00,0.00}{\textbf{#1}}}
\newcommand{\ExtensionTok}[1]{#1}
\newcommand{\FloatTok}[1]{\textcolor[rgb]{0.00,0.00,0.81}{#1}}
\newcommand{\FunctionTok}[1]{\textcolor[rgb]{0.13,0.29,0.53}{\textbf{#1}}}
\newcommand{\ImportTok}[1]{#1}
\newcommand{\InformationTok}[1]{\textcolor[rgb]{0.56,0.35,0.01}{\textbf{\textit{#1}}}}
\newcommand{\KeywordTok}[1]{\textcolor[rgb]{0.13,0.29,0.53}{\textbf{#1}}}
\newcommand{\NormalTok}[1]{#1}
\newcommand{\OperatorTok}[1]{\textcolor[rgb]{0.81,0.36,0.00}{\textbf{#1}}}
\newcommand{\OtherTok}[1]{\textcolor[rgb]{0.56,0.35,0.01}{#1}}
\newcommand{\PreprocessorTok}[1]{\textcolor[rgb]{0.56,0.35,0.01}{\textit{#1}}}
\newcommand{\RegionMarkerTok}[1]{#1}
\newcommand{\SpecialCharTok}[1]{\textcolor[rgb]{0.81,0.36,0.00}{\textbf{#1}}}
\newcommand{\SpecialStringTok}[1]{\textcolor[rgb]{0.31,0.60,0.02}{#1}}
\newcommand{\StringTok}[1]{\textcolor[rgb]{0.31,0.60,0.02}{#1}}
\newcommand{\VariableTok}[1]{\textcolor[rgb]{0.00,0.00,0.00}{#1}}
\newcommand{\VerbatimStringTok}[1]{\textcolor[rgb]{0.31,0.60,0.02}{#1}}
\newcommand{\WarningTok}[1]{\textcolor[rgb]{0.56,0.35,0.01}{\textbf{\textit{#1}}}}
\usepackage{graphicx}
\makeatletter
\def\maxwidth{\ifdim\Gin@nat@width>\linewidth\linewidth\else\Gin@nat@width\fi}
\def\maxheight{\ifdim\Gin@nat@height>\textheight\textheight\else\Gin@nat@height\fi}
\makeatother
% Scale images if necessary, so that they will not overflow the page
% margins by default, and it is still possible to overwrite the defaults
% using explicit options in \includegraphics[width, height, ...]{}
\setkeys{Gin}{width=\maxwidth,height=\maxheight,keepaspectratio}
% Set default figure placement to htbp
\makeatletter
\def\fps@figure{htbp}
\makeatother
\setlength{\emergencystretch}{3em} % prevent overfull lines
\providecommand{\tightlist}{%
  \setlength{\itemsep}{0pt}\setlength{\parskip}{0pt}}
\setcounter{secnumdepth}{-\maxdimen} % remove section numbering
\usepackage{fancyhdr}
\pagestyle{fancy}
\fancyhead[CO,CE]{Rodriguez Castro, Raul V01030827}
\ifLuaTeX
  \usepackage{selnolig}  % disable illegal ligatures
\fi
\IfFileExists{bookmark.sty}{\usepackage{bookmark}}{\usepackage{hyperref}}
\IfFileExists{xurl.sty}{\usepackage{xurl}}{} % add URL line breaks if available
\urlstyle{same}
\hypersetup{
  pdftitle={STAT 260 R Assignment 2},
  pdfauthor={Rodriguez Castro Raul V01030827},
  hidelinks,
  pdfcreator={LaTeX via pandoc}}

\title{STAT 260 R Assignment 2}
\author{Rodriguez Castro Raul V01030827}
\date{}

\begin{document}
\maketitle

P(X = 2) -\textgreater{} dbinom(2, size=18, prob=0.171) P(X ≤ 3)
-\textgreater{} pbinom(3, size=18, prob=0.171)

\hypertarget{question-1}{%
\section{Question 1}\label{question-1}}

\begin{Shaded}
\begin{Highlighting}[]
\NormalTok{lambda }\OtherTok{=} \FloatTok{4.5} \SpecialCharTok{*} \FloatTok{7.5} \CommentTok{\#seconds * duration}
\end{Highlighting}
\end{Shaded}

\#\# Section a P(X \textless= 35) i.e cdf

\begin{Shaded}
\begin{Highlighting}[]
  \FunctionTok{ppois}\NormalTok{(}\AttributeTok{q =} \DecValTok{35}\NormalTok{, }\AttributeTok{lambda =}\NormalTok{ lambda)}
\end{Highlighting}
\end{Shaded}

\begin{verbatim}
## [1] 0.6282507
\end{verbatim}

\#\# Section b

P(X = 33) i.e pmf (discrete) or pdf (continuous)

\begin{Shaded}
\begin{Highlighting}[]
  \FunctionTok{dpois}\NormalTok{(}\DecValTok{33}\NormalTok{, lambda)}
\end{Highlighting}
\end{Shaded}

\begin{verbatim}
## [1] 0.06869264
\end{verbatim}

\#\# Section c P(30 \textless= X \textless= 36) = P(X \textless= 36) -
P(X \textless= 29) scale down to sample space

\begin{Shaded}
\begin{Highlighting}[]
\FunctionTok{ppois}\NormalTok{(}\DecValTok{36}\NormalTok{, lambda) }\SpecialCharTok{{-}} \FunctionTok{ppois}\NormalTok{(}\DecValTok{29}\NormalTok{, lambda)}
\end{Highlighting}
\end{Shaded}

\begin{verbatim}
## [1] 0.4536192
\end{verbatim}

\hypertarget{question-2}{%
\section{Question 2}\label{question-2}}

\begin{Shaded}
\begin{Highlighting}[]
\NormalTok{  blades\_total }\OtherTok{=} \DecValTok{196}
\NormalTok{  blades\_prob }\OtherTok{=} \FloatTok{0.11}
\end{Highlighting}
\end{Shaded}

\#\# Section a

\begin{Shaded}
\begin{Highlighting}[]
  \CommentTok{\# P(X \textgreater{}= 20) = 1 {-} P(X \textless{}= 19)}

\NormalTok{  at\_least\_20 }\OtherTok{=} \DecValTok{1} \SpecialCharTok{{-}} \FunctionTok{pbinom}\NormalTok{(}\DecValTok{19}\NormalTok{, }\AttributeTok{size =}\NormalTok{ blades\_total, }\AttributeTok{prob =}\NormalTok{ blades\_prob)}

  \CommentTok{\#P(X = 20)}
\NormalTok{  equals\_20 }\OtherTok{=} \FunctionTok{dbinom}\NormalTok{(}\DecValTok{20}\NormalTok{, }\AttributeTok{size =}\NormalTok{ blades\_total, }\AttributeTok{prob =}\NormalTok{ at\_least\_20)}
\end{Highlighting}
\end{Shaded}

\#\# Section b

\begin{Shaded}
\begin{Highlighting}[]
\NormalTok{x }\OtherTok{=} \FunctionTok{seq}\NormalTok{(}\DecValTok{0}\NormalTok{,}\DecValTok{50}\NormalTok{, }\AttributeTok{by =} \DecValTok{1}\NormalTok{)}
\NormalTok{y }\OtherTok{=} \FunctionTok{dbinom}\NormalTok{(x, }\AttributeTok{size =}\NormalTok{ blades\_total, }\AttributeTok{prob =} \FloatTok{0.11}\NormalTok{)}

\FunctionTok{plot}\NormalTok{(x, y, }\AttributeTok{type =} \StringTok{\textquotesingle{}h\textquotesingle{}}\NormalTok{,}
     \AttributeTok{main =} \StringTok{\textquotesingle{}Title\textquotesingle{}}\NormalTok{,}
     \AttributeTok{xlab =} \StringTok{\textquotesingle{}X title\textquotesingle{}}
\NormalTok{     )}
\FunctionTok{grid}\NormalTok{()}
\end{Highlighting}
\end{Shaded}

\includegraphics{assignment-2_files/figure-latex/2b-1.pdf} \#\# Section
c

\begin{Shaded}
\begin{Highlighting}[]
  \CommentTok{\#(P X \textgreater{}= 20) = 1 {-} P(X \textgreater{}= 19)}
  
\NormalTok{  s }\OtherTok{=} \FunctionTok{sqrt}\NormalTok{(blades\_total }\SpecialCharTok{*}\NormalTok{ blades\_prob }\SpecialCharTok{*}\NormalTok{ (}\DecValTok{1} \SpecialCharTok{{-}}\NormalTok{ blades\_prob)) }\CommentTok{\# sqrt(npq)}
\NormalTok{  mu }\OtherTok{=}\NormalTok{ blades\_total }\SpecialCharTok{*}\NormalTok{ blades\_prob}

\NormalTok{  approximation }\OtherTok{=} \DecValTok{1} \SpecialCharTok{{-}} \FunctionTok{pnorm}\NormalTok{(}\DecValTok{19}\NormalTok{, }\AttributeTok{mean =}\NormalTok{ mu, }\AttributeTok{sd =}\NormalTok{ s)}
\NormalTok{  approximation}
\end{Highlighting}
\end{Shaded}

\begin{verbatim}
## [1] 0.7205291
\end{verbatim}

\begin{Shaded}
\begin{Highlighting}[]
  \CommentTok{\#1 {-} pbinom(19, size = blades\_total, prob = blades\_prob)}
\end{Highlighting}
\end{Shaded}

\#\# Section d

\begin{Shaded}
\begin{Highlighting}[]
\NormalTok{    random\_X }\OtherTok{=} \FunctionTok{dnorm}\NormalTok{(x, }\AttributeTok{mean =}\NormalTok{ mu, }\AttributeTok{sd =}\NormalTok{ s)}
    \FunctionTok{plot}\NormalTok{(x, random\_X, }\AttributeTok{type =} \StringTok{\textquotesingle{}h\textquotesingle{}}\NormalTok{,}
     \AttributeTok{main =} \StringTok{\textquotesingle{}Probability They Get Replaced\textquotesingle{}}\NormalTok{,}
     \AttributeTok{xlab =} \StringTok{\textquotesingle{}number of blades\textquotesingle{}}\NormalTok{)}
\end{Highlighting}
\end{Shaded}

\includegraphics{assignment-2_files/figure-latex/2d-1.pdf}

\hypertarget{question-3}{%
\section{Question 3}\label{question-3}}

\#\# Section a

P(23.7 \textless= X \textless= 30.4) = P(X \textless= 30.4) - P(X
\textless= 23.7) note self: continuous is inclusive P(X \textless= 23.7)

\begin{Shaded}
\begin{Highlighting}[]
\NormalTok{  mean }\OtherTok{=} \FloatTok{28.3}
\NormalTok{  sd }\OtherTok{=} \FloatTok{2.39}
\NormalTok{  x }\OtherTok{=}  \FloatTok{23.7}
  
  \FunctionTok{pnorm}\NormalTok{(}\FloatTok{30.4}\NormalTok{, }\AttributeTok{mean =}\NormalTok{ mean, }\AttributeTok{sd =}\NormalTok{ sd) }\SpecialCharTok{{-}} \FunctionTok{pnorm}\NormalTok{(x, }\AttributeTok{mean =}\NormalTok{ mean, }\AttributeTok{sd =}\NormalTok{ sd)}
\end{Highlighting}
\end{Shaded}

\begin{verbatim}
## [1] 0.7830732
\end{verbatim}

\#\# Section b

P(X \textgreater= 27.4) = 1- P(X \textless= 27.3)

\begin{Shaded}
\begin{Highlighting}[]
  \DecValTok{1} \SpecialCharTok{{-}} \FunctionTok{pnorm}\NormalTok{(}\FloatTok{27.4}\NormalTok{, }\AttributeTok{mean =}\NormalTok{ mean, }\AttributeTok{sd =}\NormalTok{ sd)}
\end{Highlighting}
\end{Shaded}

\begin{verbatim}
## [1] 0.646753
\end{verbatim}

\#\# Section c

P(25 \textless= X \textless= 31.6) = P(X \textless= 31.6) - P(X
\textless= 25)

\begin{Shaded}
\begin{Highlighting}[]
  \FunctionTok{pnorm}\NormalTok{(}\FloatTok{31.6}\NormalTok{, }\AttributeTok{mean =}\NormalTok{ mean, }\AttributeTok{sd =}\NormalTok{ sd) }\SpecialCharTok{{-}} \FunctionTok{pnorm}\NormalTok{(}\DecValTok{25}\NormalTok{, }\AttributeTok{mean =}\NormalTok{ mean, }\AttributeTok{sd =}\NormalTok{ sd)}
\end{Highlighting}
\end{Shaded}

\begin{verbatim}
## [1] 0.8326451
\end{verbatim}

\#\# Section d

\begin{Shaded}
\begin{Highlighting}[]
  \FunctionTok{qnorm}\NormalTok{(}\FloatTok{0.35}\NormalTok{, }\AttributeTok{mean =}\NormalTok{ mean, }\AttributeTok{sd =}\NormalTok{ sd)}
\end{Highlighting}
\end{Shaded}

\begin{verbatim}
## [1] 27.37908
\end{verbatim}

\hypertarget{question-4}{%
\section{Question 4}\label{question-4}}

\#\# Section a

\begin{Shaded}
\begin{Highlighting}[]
  \FunctionTok{set.seed}\NormalTok{(}\DecValTok{111}\NormalTok{)}
\NormalTok{  simulation.data }\OtherTok{=} \FunctionTok{rbinom}\NormalTok{(}\DecValTok{2800}\NormalTok{, }\AttributeTok{size =} \DecValTok{72}\NormalTok{, }\AttributeTok{prob =} \FloatTok{0.36}\NormalTok{)}
\end{Highlighting}
\end{Shaded}

\#\# Section b

\begin{Shaded}
\begin{Highlighting}[]
  \FunctionTok{hist}\NormalTok{(simulation.data, }
     \AttributeTok{main =} \StringTok{"Histogram of Simulation Data"}\NormalTok{,}
     \AttributeTok{xlab =} \StringTok{"Values of X"}\NormalTok{,}
     \AttributeTok{ylab =} \StringTok{"Frequency"}\NormalTok{,}
     \AttributeTok{col =} \StringTok{"skyblue"}\NormalTok{,  }
     \AttributeTok{breaks =} \DecValTok{20}\NormalTok{) }
\end{Highlighting}
\end{Shaded}

\includegraphics{assignment-2_files/figure-latex/4b-1.pdf} Histogram is
normally distributed with a slight right skew.

\#\# Section c

\begin{Shaded}
\begin{Highlighting}[]
  \FunctionTok{mean}\NormalTok{(simulation.data)}
\end{Highlighting}
\end{Shaded}

\begin{verbatim}
## [1] 25.87071
\end{verbatim}

The simulation is very close to the expected value indicating that the
sample mean is a reliable estimate of the population mean.

\end{document}
